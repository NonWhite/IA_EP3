\section{Configuração de experimentos}
\label{sec:setup}

Para todos os experimentos foi usada a livraria scikit-learn (Python)\footnote{Official Site: http://scikit-learn.org/stable/}. Além disso, para cada classificador foram usadas as seguinte configurações:
\begin{itemize}
	\item K-Nearest Neighbors
	\begin{itemize}
		\item metric: euclidean, manhattan e hamming. Função da distancia entre vizinhos
		\item k: 1, 5 , 10 e 50. Número de vizinhos a considerar por ponto
	\end{itemize}
	\item Decision Tree
	\begin{itemize}
		\item criterion: gini e entropy. Critério para seleção de atributos
		\item depth: 10, 50 e 100. Máxima profundidade da árvore
	\end{itemize}
	\item Naive Bayes
	\begin{itemize}
		\item alpha: 1.0 , 1.0 , 100.0. Constante para Laplace smoothing
	\end{itemize}
\end{itemize}

Os resultados para cada uma das configurações serão mostrados nas seções~\ref{sec:knn},~\ref{sec:decision} e~\ref{sec:bayes} usando só o conjunto de treinamento com validação cruzada. Na seção~\ref{sec:validation}, as melhores configurações serão usadas com ambos conjuntos (treinamento e validação) para encontrar o melhor classificador para o domínio.