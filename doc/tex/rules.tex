\section{Regras}
\label{sec:rules}

Além das características que descrevem cada personagem, a base de conhecimento precisa de regras para poder inferir qual é a personagem que o usuário está pensando. Nesta seção serão explicadas as regras colocadas na base de conhecimento.

\subsection{Regras de descrição}
\label{subsec:description}
Para que o programa consiga inferir o personagem, tem que existir fatos diferentes na base de conhecimento para cada um. Então para cada característica o jeito de representar um fato é da seguinte forma:
	\begin{itemize}
		\item $V$\_type
		\item $V$\_color
		\item $V$\_size
		\item $V$\_weight
	\end{itemize}
Onde $V$ é um dos possíveis valores para cada atributo. Além disso, para as características binarias (true,false) só vão existir se fossem verdadeiros para algum personagem.\\
Por exemplo, a primeira personagem se representa
	\begin{lstlisting}[ caption = Bulbasaur , label = code:bulbasaur ]
	bulbasaur :- verify( grass_type ) ,
				verify( poison_type ) ,
				verify( green_color ) ,
				verify( small_size ) ,
				verify( light_weight ) ,
				verify( has_evolution ) ,
				verify( is_starter ) , ! .
	\end{lstlisting}
Por último, a função ${verify}$ é usada para perguntar ao usuário se aquele fato é verdade para a personagem que ele está pensando.

\subsection{Regras de exclusão}
\label{subsec:exclusion}

Se só consideramos as regras de descrição anteriores, o programa teria que fazer muitas perguntas para encontrar qual é a personagem do usuário, mas muitos valores das características de uma personagem são exclusivos. Por exemplo, uma personagem não pode ter tamanho pequeno e tamanho grande ao mesmo tempo, do mesmo jeito, não pode ter como cor principal verde e azul ao mesmo tempo (cada personagem tem só um cor principal).\\
Então foram adicionadas regras de exclusão em Prolog que tem a seguinte forma:
	\[ no( V_k ) :- yes( V_2 ) ; yes( V_3 ) ; \ldots ; yes( V_{k-1} ) ; yes( V_{k+1} ) ; \ldots ; yes( V_n ) . \]
onde $V_i$ é um dos possíveis valores de uma característica. O que aquela regra diz é que para algum valor $k$ de uma característica, se algum dos fatos para os outros valores é verdadeiro, então não pode ser verdadeiro o valor $k$ também.

Por exemplo, para a característica Size temos \footnote{O operador lógico OR pode ser escrito como $";"$ em Prolog ou pode ser colocado implicitamente em varias linhas como o exemplo.}:
	\begin{lstlisting}
		no( small_size ) :- yes( big_size ) , !.
		no( small_size ) :- yes( medium_size ) , !.
		no( big_size ) :- yes( small_size ) , !.
		no( big_size ) :- yes( medium_size ) , !.
		no( medium_size ) :- yes( small_size ) , !.
		no( medium_size ) :- yes( big_size ) , !.
	\end{lstlisting}

Um caso especial de estas regras é feito para a característica Type porque uma personagem pode ter mais de uma e então não todas as possíveis combinações de pares geram regras de exclusão. No exemplo~\ref{code:bulbasaur}, a personagem é do tipo grass e poison e então na base de conhecimento não temos uma regra de exclusão para aquele par de valores.